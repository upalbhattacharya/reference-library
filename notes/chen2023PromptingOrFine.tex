\documentclass[a4paper,colorinlistoftodos]{article}
%%%%%%%%%%%%%%%%%%%%%%%%%%%%%%%%%%%%%%%%%%%%%%%%%%%%%%%%%%%%%%%%%%%%%%
% Packages, Macros and Commands
%%%%%%%%%%%%%%%%%%%%%%%%%%%%%%%%%%%%%%%%%%%%%%%%%%%%%%%%%%%%%%%%%%%%%%

%%%%%%%%%% Basic %%%%%%%%%%
\usepackage[utf8]{inputenc}
\usepackage[T1]{fontenc}
\usepackage[displaymath,mathlines,running]{lineno}

%%%%%%%%%% Referencing %%%%%%%%%%
\usepackage[numbers,sort&compress]{natbib}

%%%%%%%%%% Images %%%%%%%%%%
\usepackage{graphicx}
\usepackage{subcaption}      % For subfigures

%%%%%%%%%% Formatting %%%%%%%%%%
\usepackage{longtable}
\usepackage[hidelinks]{hyperref}
\hypersetup{
  colorlinks = true,
  linkcolor = cat-latte-red}

%%%%%%%%%% Math %%%%%%%%%%
\usepackage{amsmath}
\usepackage{amssymb}
\usepackage{amsthm}

%%%%%%%%%% Presentation %%%%%%%%%%
\usepackage[dvipsnames]{xcolor}
\usepackage[normalem]{ulem}
\usepackage{tcolorbox}
\usepackage{soul}

%%%%%%%%%% Misc %%%%%%%%%%
\usepackage{todonotes}

\newcounter{todocounter}
\newcommand{\todonum}[2][]
{\stepcounter{todocounter}\todo[#1]{\thetodocounter: #2}}

\usepackage[commandnameprefix=always]{changes}
\definechangesauthor[name='Upal Bhattacharya', color=cat-latte-lavendar]{UB}

%%%%%%%%%%%%%%%%%%%%%%%%%%%%%%%%%%%%%%%%%%%%%%%%%%%%%%%%%%%%%%%%%%%%%%
% Customization
%%%%%%%%%%%%%%%%%%%%%%%%%%%%%%%%%%%%%%%%%%%%%%%%%%%%%%%%%%%%%%%%%%%%%%
\tcbuselibrary{theorems}

%%%%%%%%%% Colors (Catppuccin Latte) %%%%%%%%%%%
\definecolor{cat-latte-green}{HTML}{40A02B}
\definecolor{cat-latte-orange}{HTML}{FE640B}
\definecolor{cat-latte-yellow}{HTML}{DF8E1D}
\definecolor{cat-latte-red}{HTML}{D20F39}
\definecolor{cat-latte-blue}{HTML}{04A5E5}
\definecolor{cat-latte-gray}{HTML}{7c7f93}
\definecolor{cat-latte-purple}{HTML}{8839EF}
\definecolor{cat-latte-lavendar}{HTML}{7287fd}

\colorlet{soulred}{cat-latte-red!50}
\colorlet{soulgreen}{cat-latte-green!50}
\colorlet{soulyellow}{cat-latte-yellow!50}
\colorlet{soulorange}{cat-latte-orange!50}
\colorlet{soulblue}{cat-latte-blue!50}
\colorlet{soulgray}{cat-latte-gray!50}

%%%%%%%%%% Callouts %%%%%%%%%%
\newtcbtheorem[auto counter]{caution}{Caution}%
{colback=cat-latte-orange!15,colframe=cat-latte-orange,fonttitle=\bfseries}{caution}
\newtcbtheorem[auto counter]{positive}{Positive}%
{colback=cat-latte-green!15,colframe=cat-latte-green,fonttitle=\bfseries}{positive}
\newtcbtheorem[auto counter]{query}{Query}%
{colback=cat-latte-yellow!15,colframe=cat-latte-yellow,fonttitle=\bfseries}{query}
\newtcbtheorem[auto counter]{answer}{Answer}%
{colback=cat-latte-blue!15,colframe=cat-latte-blue,fonttitle=\bfseries}{answer}
\newtcbtheorem[auto counter]{negative}{Negative}%
{colback=cat-latte-red!15,colframe=cat-latte-red,fonttitle=\bfseries}{negative}
\newtcbtheorem[auto counter]{gap}{Gap}%
{colback=cat-latte-gray!15,colframe=cat-latte-gray,fonttitle=\bfseries}{gap}
\newtcbtheorem[no counter]{upal}{Upal}%
{colback=cat-latte-purple!15,colframe=cat-latte-purple,fonttitle=\bfseries}{upal}
\newtcbtheorem[no counter]{project}{Project}%
{colback=cat-latte-lavendar!15,colframe=cat-latte-lavendar,fonttitle=\bfseries}{project}

%%%%%%%%%% Highlights %%%%%%%%%%
\newcommand{\hlred}[1]{\sethlcolor{soulred}\hl{#1}}
\newcommand{\hlgreen}[1]{\sethlcolor{soulgreen}\hl{#1}}
\newcommand{\hlyellow}[1]{\sethlcolor{soulyellow}\hl{#1}}
\newcommand{\hlblue}[1]{\sethlcolor{soulblue}\hl{#1}}
\newcommand{\hlorange}[1]{\sethlcolor{soulorange}\hl{#1}}
\newcommand{\hlgray}[1]{\sethlcolor{soulgray}\hl{#1}}

%%%%%%%%%% Underline %%%%%%%%%%
\setul{}{1pt} % Width
\newcommand{\ulred}[1]{\setulcolor{cat-latte-red}\ul{#1}}
\newcommand{\ulgreen}[1]{\setulcolor{cat-latte-green}\ul{#1}}
\newcommand{\ulyellow}[1]{\setulcolor{cat-latte-yellow}\ul{#1}}
\newcommand{\ulblue}[1]{\setulcolor{cat-latte-blue}\ul{#1}}
\newcommand{\ulorange}[1]{\setulcolor{cat-latte-orange}\ul{#1}}
\newcommand{\ulgray}[1]{\setulcolor{cat-latte-gray}\ul{#1}}

%%%%%%%%%%%%%%%%%%%%%%%%%%%%%%%%%%%%%%%%%%%%%%%%%%%%%%%%%%%%%%%%%%%%%%
% Project-Specific
%%%%%%%%%%%%%%%%%%%%%%%%%%%%%%%%%%%%%%%%%%%%%%%%%%%%%%%%%%%%%%%%%%%%%%

%%%%%%%%%% Imports and Commands %%%%%%%%%%

%%%%%%%%%% Customization %%%%%%%%%%

\usepackage{environ}

\newif\ifhide
\hidetrue % toggle if necessary

% Callouts
\ifhide
  \NewEnviron{hide}{}
  \let\caution\hide
  \let\endcaution\endhide
  
  \let\positive\hide
  \let\endpositive\endhide
  
  \let\negative\hide
  \let\endnegative\endhide
  
  \let\gap\hide
  \let\endgap\endhide
  
  \let\upal\hide
  \let\endupal\endhide
  
  \let\query\hide
  \let\endquery\endhide
  
  \let\answer\hide
  \let\endanswer\endhide
  
  \let\project\hide
  \let\endproject\endhide
\fi

% Highlights
\renewcommand\hlred[1]{#1}
\renewcommand\hlgreen[1]{#1}
\renewcommand\hlblue[1]{#1}
\renewcommand\hlyellow[1]{#1}
\renewcommand\hlorange[1]{#1}
\renewcommand\hlgray[1]{#1}

% Underline
\renewcommand\ulred[1]{#1}
\renewcommand\ulgreen[1]{#1}
\renewcommand\ulblue[1]{#1}
\renewcommand\ulyellow[1]{#1}
\renewcommand\ulorange[1]{#1}
\renewcommand\ulgray[1]{#1}

% To hide todonotes, use: \usepackage[disable]{todonotes} in preamble.tex
% To hide changes, use: \usepackage[final]{changes} in preamble.tex


% \author{Upal Bhattacharya}
\date{}
\title{Notes for ``Prompting or Fine-tuning? A Comparative Study of
 Large Language Models for Taxonomy Construction''}
\begin{document}

\maketitle

\begingroup
    \hypersetup{linkcolor=black}
    \tableofcontents
    \pagebreak
\endgroup

\linenumbers

\section{Summary}
\label{sec:summary}

The paper presents a study to compare the performance of fine-tuned LLMs
vs. prompting LLMs for structurally consistent taxonomy construction.
Two datasets are constructed from popular data sources and, using various
fine-tuning strategies (layer-wise, LoRA) and taxonomy construction objectives
(MALI, MSA, MV), performance by prompting is found to be better than that
obtained by fine-tuning. The study does not explore the generality of this
claim by extending their study to include low-resource datasets
e.g. CASE. While highlighting structurally consistent methods as one
evaluation criteria, the analysis over strcutural consistency is entirely
lacking. Further, more structurally inconsistent techniques (post-processing
strategies) are used with only one structurally consistent method (fine-tuning
using MSA). Structural consistency is therefore not explored well. The `best'
performer, despite being mildly structurally inconsistent, is considered the
best. This is confusing given the importance initially laid on structural
consistency.

The study highlights how prompting can provide better results than fine-tuning
for taxonomy construction. However, it misses out on verifying this with
low-resource datasets and creates confusion by emphasising structural
consistency and then touting a mildly structurally inconsistent method as the
better performer.

\section{Brief Notes}
\label{sec:brief-notes}

\begin{itemize}
\item Compares performance of fine-tuned models for taxonomy construction
  vs. few-shot prompting using different maximization objectives for
  constructing the final taxonomy to find prompting methods performing better.
\item Uses a two-module system:
  \begin{itemize}
  \item \textbf{Relation Prediction}: Predicts relations (as scores) from  concept
    pairs.
  \item \textbf{Post-processing}: Uses scores and either a \textit{maximum likelihood}
    objective (that does not conform to structural constraints) or a
    \textit{maximum arborescence} objective (conforms to structural constraints)
    to select a subgraph from the dense graph of all relations given by the
    relation prediction module to construct a taxonomy.
  \end{itemize}
  \begin{definition}{Arborescence}{def:arborescence}
    An arborescence is a directed graph where there exists a vertex $r$ (root)
    such that, for any other vertex $v$, there is exactly one directed walk
    from $r$ to $v$.
  \end{definition}
\end{itemize}

\subsection{Methodology}
\label{subsec:methodology}

\begin{itemize}
\item \textbf{Dataset}: Compares fine-tuning vs. prompting performance on a
  medium-sized variant of WordNet and a new split-up variant of ACM CCS. 13
  original taxonomies of the ACM CCS are split up into 75 sub-taxonomies to
  account for the size imbalance in the original taxonomies by splitting at
  level 1 of the taxonomy trees and accounting for duplication removal (144).
  Datasets constructed using a 70-20-10 split including all positive relation
  concept pairs and randomly selected negative pairs from each concept.
  \begin{caution}{Artificial splitting of ACM CCS}{caut-1}
    Artificially splitting the taxonomy deviates from evaluation over
    realistic conditions. Real taxonomies are varied in size and
    structure. The sub-taxonomies created from the split may be varied in
    size as well but some homogenizing factor (at least 70 terms?) and the
    lack of reported structural differences in the new sub-trees makes it
    difficult to assess performance variability over structural differences.
  \end{caution}
  \begin{positive}{Difficulty in structural difference-based
      analysis}{pos-1}
    The omission of structural metrics of the hyponym taxonomy and of the
    newly constructed ACM CCS taxonomy suggests that it may not be simple to
    reason over differences in performance arising from structure. This is
    similar to the lack of depth in structural variation analysis in our
    first research study.
  \end{positive}
\item \textbf{Fine-tuning Setup}: GPT-Neo trained using layer-wise or LoRA
  using post-processing objectives (MALI, MSA) and a pairwise classification
  task. For a concept pair $(A, B)$, a sentence template: \newline
  \verb|I am doing the taxonomy research, I think A is a subtopic of B| is
  used to create the input to the model for fine-tuning.
\item \textbf{Prompting Setup}: GPT-3.5-turbo is prompted to generate all
  possible relations along with scores using a single prompt. A singular
  prompt with all concept labels and randomly selected illustrative examples
  of ENTIRE TAXONOMIES (5 for WordNet; 3 for ACM CCS) is given with the
  objective of generating all possible relations and their
  scores(?). Randomness is accounted for by mutiple repetitions and using a
  \textbf{majority voting post-processing} objective to obtain the final
  taxonomy (does not respect structural constraints)
\item \textbf{Performance Comparison}: For the fine-tuning approaches,
  compared over layer-wise vs. LoRA and MALI vs. MSA. Prompting methods
  compared over Best vs majority voting. Evaluated over Precision, Recall and
  F1.
\end{itemize}
\subsection{Results}
\label{subsec:results}

\begin{itemize}
\item Comparing with the ground truth, prompting approaches outperform the
  fine-tuning methods. Although slightly structurally inconsistent, they are
  still pretty good. Only MSA fine-tuning methods are structurally consistent.

  \begin{negative}{Relevance of structural consistency}{neg-1}
    The paper considers structural consistency important but never gets into
    what structural inconsistencies are generated by the prompting and
    MALI-based approaches. While the prompting methods appear almost
    strcuturally consistent (by the metrics they highligh in Table II), it is
    difficult to understand the relevance of this without further analysis.
  \end{negative}
\item Acknowledges that present approach relies on semantic meaningfulness of
  concept labels and the alignment of their semantics with the semantics of
  tokens as learnt by LLMs.
  \begin{negative}{Promting vs. fine-tuning on low-resource datasets}{neg-2}
    The contributions of the paper suggest a general statement that prompting
    can outperform fine-tuning strategies for taxonomy construction. However,
    the considered taxonomies are surely very well understood by the LLMs from
    their pre-training data. Hyponym classification is basically trained from
    all English resources while ACM CCS is undoubtedly in the training
    corpus. As a result, the generality of the statement is incorrect. The
    prompting methods outperforming fine-tuning relies on the implicit
    knowledge they already possess of these entities wherein, it is possible
    that the fine-tuning strategies actually causes GPT-Neo to unlearn
    this. Evaluation over low-resource datasets alongside `popular' data
    sources would provide a better overview of performance.
    

\end{negative}


\end{itemize}


\bibliographystyle{splncs04nat}
\bibliography{bibliography}

\pagebreak
\nolinenumbers
\appendix

\begingroup
    \hypersetup{linkcolor=black}
    \listoftodos
    \listofchanges
    \pagebreak
\endgroup

\end{document}

