\documentclass[a4paper,colorinlistoftodos]{article}
%%%%%%%%%%%%%%%%%%%%%%%%%%%%%%%%%%%%%%%%%%%%%%%%%%%%%%%%%%%%%%%%%%%%%%
% Packages, Macros and Commands
%%%%%%%%%%%%%%%%%%%%%%%%%%%%%%%%%%%%%%%%%%%%%%%%%%%%%%%%%%%%%%%%%%%%%%

%%%%%%%%%% Basic %%%%%%%%%%
\usepackage[utf8]{inputenc}
\usepackage[T1]{fontenc}
\usepackage[displaymath,mathlines,running]{lineno}

%%%%%%%%%% Referencing %%%%%%%%%%
\usepackage[numbers,sort&compress]{natbib}

%%%%%%%%%% Images %%%%%%%%%%
\usepackage{graphicx}
\usepackage{subcaption}      % For subfigures

%%%%%%%%%% Formatting %%%%%%%%%%
\usepackage{longtable}
\usepackage[hidelinks]{hyperref}
\hypersetup{
  colorlinks = true,
  linkcolor = cat-latte-red}

%%%%%%%%%% Math %%%%%%%%%%
\usepackage{amsmath}
\usepackage{amssymb}
\usepackage{amsthm}

%%%%%%%%%% Presentation %%%%%%%%%%
\usepackage[dvipsnames]{xcolor}
\usepackage[normalem]{ulem}
\usepackage{tcolorbox}
\usepackage{soul}

%%%%%%%%%% Misc %%%%%%%%%%
\usepackage{todonotes}

\newcounter{todocounter}
\newcommand{\todonum}[2][]
{\stepcounter{todocounter}\todo[#1]{\thetodocounter: #2}}

\usepackage[commandnameprefix=always]{changes}
\definechangesauthor[name='Upal Bhattacharya', color=cat-latte-lavendar]{UB}

%%%%%%%%%%%%%%%%%%%%%%%%%%%%%%%%%%%%%%%%%%%%%%%%%%%%%%%%%%%%%%%%%%%%%%
% Customization
%%%%%%%%%%%%%%%%%%%%%%%%%%%%%%%%%%%%%%%%%%%%%%%%%%%%%%%%%%%%%%%%%%%%%%
\tcbuselibrary{theorems}

%%%%%%%%%% Colors (Catppuccin Latte) %%%%%%%%%%%
\definecolor{cat-latte-green}{HTML}{40A02B}
\definecolor{cat-latte-orange}{HTML}{FE640B}
\definecolor{cat-latte-yellow}{HTML}{DF8E1D}
\definecolor{cat-latte-red}{HTML}{D20F39}
\definecolor{cat-latte-blue}{HTML}{04A5E5}
\definecolor{cat-latte-gray}{HTML}{7c7f93}
\definecolor{cat-latte-purple}{HTML}{8839EF}
\definecolor{cat-latte-lavendar}{HTML}{7287fd}

\colorlet{soulred}{cat-latte-red!50}
\colorlet{soulgreen}{cat-latte-green!50}
\colorlet{soulyellow}{cat-latte-yellow!50}
\colorlet{soulorange}{cat-latte-orange!50}
\colorlet{soulblue}{cat-latte-blue!50}
\colorlet{soulgray}{cat-latte-gray!50}

%%%%%%%%%% Callouts %%%%%%%%%%
\newtcbtheorem[auto counter]{caution}{Caution}%
{colback=cat-latte-orange!15,colframe=cat-latte-orange,fonttitle=\bfseries}{caution}
\newtcbtheorem[auto counter]{positive}{Positive}%
{colback=cat-latte-green!15,colframe=cat-latte-green,fonttitle=\bfseries}{positive}
\newtcbtheorem[auto counter]{query}{Query}%
{colback=cat-latte-yellow!15,colframe=cat-latte-yellow,fonttitle=\bfseries}{query}
\newtcbtheorem[auto counter]{answer}{Answer}%
{colback=cat-latte-blue!15,colframe=cat-latte-blue,fonttitle=\bfseries}{answer}
\newtcbtheorem[auto counter]{negative}{Negative}%
{colback=cat-latte-red!15,colframe=cat-latte-red,fonttitle=\bfseries}{negative}
\newtcbtheorem[auto counter]{gap}{Gap}%
{colback=cat-latte-gray!15,colframe=cat-latte-gray,fonttitle=\bfseries}{gap}
\newtcbtheorem[no counter]{upal}{Upal}%
{colback=cat-latte-purple!15,colframe=cat-latte-purple,fonttitle=\bfseries}{upal}
\newtcbtheorem[no counter]{project}{Project}%
{colback=cat-latte-lavendar!15,colframe=cat-latte-lavendar,fonttitle=\bfseries}{project}

%%%%%%%%%% Highlights %%%%%%%%%%
\newcommand{\hlred}[1]{\sethlcolor{soulred}\hl{#1}}
\newcommand{\hlgreen}[1]{\sethlcolor{soulgreen}\hl{#1}}
\newcommand{\hlyellow}[1]{\sethlcolor{soulyellow}\hl{#1}}
\newcommand{\hlblue}[1]{\sethlcolor{soulblue}\hl{#1}}
\newcommand{\hlorange}[1]{\sethlcolor{soulorange}\hl{#1}}
\newcommand{\hlgray}[1]{\sethlcolor{soulgray}\hl{#1}}

%%%%%%%%%% Underline %%%%%%%%%%
\setul{}{1pt} % Width
\newcommand{\ulred}[1]{\setulcolor{cat-latte-red}\ul{#1}}
\newcommand{\ulgreen}[1]{\setulcolor{cat-latte-green}\ul{#1}}
\newcommand{\ulyellow}[1]{\setulcolor{cat-latte-yellow}\ul{#1}}
\newcommand{\ulblue}[1]{\setulcolor{cat-latte-blue}\ul{#1}}
\newcommand{\ulorange}[1]{\setulcolor{cat-latte-orange}\ul{#1}}
\newcommand{\ulgray}[1]{\setulcolor{cat-latte-gray}\ul{#1}}

%%%%%%%%%%%%%%%%%%%%%%%%%%%%%%%%%%%%%%%%%%%%%%%%%%%%%%%%%%%%%%%%%%%%%%
% Project-Specific
%%%%%%%%%%%%%%%%%%%%%%%%%%%%%%%%%%%%%%%%%%%%%%%%%%%%%%%%%%%%%%%%%%%%%%

%%%%%%%%%% Imports and Commands %%%%%%%%%%

%%%%%%%%%% Customization %%%%%%%%%%

\usepackage{environ}

\newif\ifhide
\hidetrue % toggle if necessary

% Callouts
\ifhide
  \NewEnviron{hide}{}
  \let\caution\hide
  \let\endcaution\endhide
  
  \let\positive\hide
  \let\endpositive\endhide
  
  \let\negative\hide
  \let\endnegative\endhide
  
  \let\gap\hide
  \let\endgap\endhide
  
  \let\upal\hide
  \let\endupal\endhide
  
  \let\query\hide
  \let\endquery\endhide
  
  \let\answer\hide
  \let\endanswer\endhide
  
  \let\project\hide
  \let\endproject\endhide
\fi

% Highlights
\renewcommand\hlred[1]{#1}
\renewcommand\hlgreen[1]{#1}
\renewcommand\hlblue[1]{#1}
\renewcommand\hlyellow[1]{#1}
\renewcommand\hlorange[1]{#1}
\renewcommand\hlgray[1]{#1}

% Underline
\renewcommand\ulred[1]{#1}
\renewcommand\ulgreen[1]{#1}
\renewcommand\ulblue[1]{#1}
\renewcommand\ulyellow[1]{#1}
\renewcommand\ulorange[1]{#1}
\renewcommand\ulgray[1]{#1}

% To hide todonotes, use: \usepackage[disable]{todonotes} in preamble.tex
% To hide changes, use: \usepackage[final]{changes} in preamble.tex


\author{Upal Bhattacharya}
\date{\today}
\title{Structured Prompt Interrogation and Recursive Extraction of Semantics (SPIRES): a method for populating knowledge bases using zero-shot learning}
\begin{document}

\maketitle

\begingroup
    \hypersetup{linkcolor=black}
    \tableofcontents
    \pagebreak
\endgroup

\linenumbers


\section{Reading TODOs}

\begin{itemize}
  \item[\faCheckSquareO] Abstract
  \item[\faCheckSquareO] Introduction
  \item[\faCheckSquareO] Conclusion
  \item[\faSquareO] Results
  \item[\faSquareO] Methods
  \item[\faSquareO] Summary
  \item[\faSquareO] Verdict
\end{itemize}

\section{Abstract + Introduction + Conclusion}
\begin{caution}{}{caution-1}
Uses Knowledge Base and ontology interchangeably. This can create
implementation confusion.
\end{caution}

\begin{itemize}
  \item Performs zero-shot recursive prompting to align LLM responses with a
    user-defined schema and an input text
    \begin{query}{}{query-1}
      Is the user-defined schema part of the prompt input or the gold standard
      against which the output is evaluated?
    \end{query}
  \item Uses existing ontologies and vocabularies to substantiate matched
    elements with identifiers
  \item Can perform new tasks without new training data
  \item \hlgreen{NLP methods can assist KB construction at several stages:
      automatic selection of relevant data,NER for entity/term/concept
      identification, RE for taxonomic and non-hierarchical relation
      extraction}
  \item Schemas define structure but complex schemas are difficult to
    populate 
  \item \hlgreen{SPIRES performs automatic population of custom schemas and
    ontology models}
\end{itemize}

\section{Methods}

\begin{itemize}
  \item A schema is a collection of classes/concepts which can be instantiated
    with instances. Each class has a list of attributes.
    \begin{query}{}{query-2}
      What is the exact form of these attributes?
    \end{query}
    An attribute has associated properties which include its name, range, etc.
   \item Models are prompted with terms from 4 ontologies with a request to
     retrieve the identifiers from the ontology.
     \begin{query}{}{query-3}
       Are identifiers like IDs e.g. GO ID? Does that not make this the same
       task as in \cite{bombieri2025DoLlmsDream}? 

    \end{query}
 
 
\end{itemize}

\section{Results}

\section{Summary}

\section{Verdict}



% \bibliographystyle{splncs04nat}
% \bibliography{/home/upal/References/bibliography.bib}
\printbibliography

\pagebreak
\nolinenumbers
\appendix

\begingroup
    \hypersetup{linkcolor=black}
    \listoftodos
    \listofchanges
    \pagebreak
\endgroup

\printbibliography
\end{document}

