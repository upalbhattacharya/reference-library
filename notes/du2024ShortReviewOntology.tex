\documentclass[a4paper,colorinlistoftodos]{article}
%%%%%%%%%%%%%%%%%%%%%%%%%%%%%%%%%%%%%%%%%%%%%%%%%%%%%%%%%%%%%%%%%%%%%%
% Packages, Macros and Commands
%%%%%%%%%%%%%%%%%%%%%%%%%%%%%%%%%%%%%%%%%%%%%%%%%%%%%%%%%%%%%%%%%%%%%%

%%%%%%%%%% Basic %%%%%%%%%%
\usepackage[utf8]{inputenc}
\usepackage[T1]{fontenc}
\usepackage[displaymath,mathlines,running]{lineno}

%%%%%%%%%% Referencing %%%%%%%%%%
\usepackage[numbers,sort&compress]{natbib}

%%%%%%%%%% Images %%%%%%%%%%
\usepackage{graphicx}
\usepackage{subcaption}      % For subfigures

%%%%%%%%%% Formatting %%%%%%%%%%
\usepackage{longtable}
\usepackage[hidelinks]{hyperref}
\hypersetup{
  colorlinks = true,
  linkcolor = cat-latte-red}

%%%%%%%%%% Math %%%%%%%%%%
\usepackage{amsmath}
\usepackage{amssymb}
\usepackage{amsthm}

%%%%%%%%%% Presentation %%%%%%%%%%
\usepackage[dvipsnames]{xcolor}
\usepackage[normalem]{ulem}
\usepackage{tcolorbox}
\usepackage{soul}

%%%%%%%%%% Misc %%%%%%%%%%
\usepackage{todonotes}

\newcounter{todocounter}
\newcommand{\todonum}[2][]
{\stepcounter{todocounter}\todo[#1]{\thetodocounter: #2}}

\usepackage[commandnameprefix=always]{changes}
\definechangesauthor[name='Upal Bhattacharya', color=cat-latte-lavendar]{UB}

%%%%%%%%%%%%%%%%%%%%%%%%%%%%%%%%%%%%%%%%%%%%%%%%%%%%%%%%%%%%%%%%%%%%%%
% Customization
%%%%%%%%%%%%%%%%%%%%%%%%%%%%%%%%%%%%%%%%%%%%%%%%%%%%%%%%%%%%%%%%%%%%%%
\tcbuselibrary{theorems}

%%%%%%%%%% Colors (Catppuccin Latte) %%%%%%%%%%%
\definecolor{cat-latte-green}{HTML}{40A02B}
\definecolor{cat-latte-orange}{HTML}{FE640B}
\definecolor{cat-latte-yellow}{HTML}{DF8E1D}
\definecolor{cat-latte-red}{HTML}{D20F39}
\definecolor{cat-latte-blue}{HTML}{04A5E5}
\definecolor{cat-latte-gray}{HTML}{7c7f93}
\definecolor{cat-latte-purple}{HTML}{8839EF}
\definecolor{cat-latte-lavendar}{HTML}{7287fd}

\colorlet{soulred}{cat-latte-red!50}
\colorlet{soulgreen}{cat-latte-green!50}
\colorlet{soulyellow}{cat-latte-yellow!50}
\colorlet{soulorange}{cat-latte-orange!50}
\colorlet{soulblue}{cat-latte-blue!50}
\colorlet{soulgray}{cat-latte-gray!50}

%%%%%%%%%% Callouts %%%%%%%%%%
\newtcbtheorem[auto counter]{caution}{Caution}%
{colback=cat-latte-orange!15,colframe=cat-latte-orange,fonttitle=\bfseries}{caution}
\newtcbtheorem[auto counter]{positive}{Positive}%
{colback=cat-latte-green!15,colframe=cat-latte-green,fonttitle=\bfseries}{positive}
\newtcbtheorem[auto counter]{query}{Query}%
{colback=cat-latte-yellow!15,colframe=cat-latte-yellow,fonttitle=\bfseries}{query}
\newtcbtheorem[auto counter]{answer}{Answer}%
{colback=cat-latte-blue!15,colframe=cat-latte-blue,fonttitle=\bfseries}{answer}
\newtcbtheorem[auto counter]{negative}{Negative}%
{colback=cat-latte-red!15,colframe=cat-latte-red,fonttitle=\bfseries}{negative}
\newtcbtheorem[auto counter]{gap}{Gap}%
{colback=cat-latte-gray!15,colframe=cat-latte-gray,fonttitle=\bfseries}{gap}
\newtcbtheorem[no counter]{upal}{Upal}%
{colback=cat-latte-purple!15,colframe=cat-latte-purple,fonttitle=\bfseries}{upal}
\newtcbtheorem[no counter]{project}{Project}%
{colback=cat-latte-lavendar!15,colframe=cat-latte-lavendar,fonttitle=\bfseries}{project}

%%%%%%%%%% Highlights %%%%%%%%%%
\newcommand{\hlred}[1]{\sethlcolor{soulred}\hl{#1}}
\newcommand{\hlgreen}[1]{\sethlcolor{soulgreen}\hl{#1}}
\newcommand{\hlyellow}[1]{\sethlcolor{soulyellow}\hl{#1}}
\newcommand{\hlblue}[1]{\sethlcolor{soulblue}\hl{#1}}
\newcommand{\hlorange}[1]{\sethlcolor{soulorange}\hl{#1}}
\newcommand{\hlgray}[1]{\sethlcolor{soulgray}\hl{#1}}

%%%%%%%%%% Underline %%%%%%%%%%
\setul{}{1pt} % Width
\newcommand{\ulred}[1]{\setulcolor{cat-latte-red}\ul{#1}}
\newcommand{\ulgreen}[1]{\setulcolor{cat-latte-green}\ul{#1}}
\newcommand{\ulyellow}[1]{\setulcolor{cat-latte-yellow}\ul{#1}}
\newcommand{\ulblue}[1]{\setulcolor{cat-latte-blue}\ul{#1}}
\newcommand{\ulorange}[1]{\setulcolor{cat-latte-orange}\ul{#1}}
\newcommand{\ulgray}[1]{\setulcolor{cat-latte-gray}\ul{#1}}

%%%%%%%%%%%%%%%%%%%%%%%%%%%%%%%%%%%%%%%%%%%%%%%%%%%%%%%%%%%%%%%%%%%%%%
% Project-Specific
%%%%%%%%%%%%%%%%%%%%%%%%%%%%%%%%%%%%%%%%%%%%%%%%%%%%%%%%%%%%%%%%%%%%%%

%%%%%%%%%% Imports and Commands %%%%%%%%%%

%%%%%%%%%% Customization %%%%%%%%%%

\usepackage{environ}

\newif\ifhide
\hidetrue % toggle if necessary

% Callouts
\ifhide
  \NewEnviron{hide}{}
  \let\caution\hide
  \let\endcaution\endhide
  
  \let\positive\hide
  \let\endpositive\endhide
  
  \let\negative\hide
  \let\endnegative\endhide
  
  \let\gap\hide
  \let\endgap\endhide
  
  \let\upal\hide
  \let\endupal\endhide
  
  \let\query\hide
  \let\endquery\endhide
  
  \let\answer\hide
  \let\endanswer\endhide
  
  \let\project\hide
  \let\endproject\endhide
\fi

% Highlights
\renewcommand\hlred[1]{#1}
\renewcommand\hlgreen[1]{#1}
\renewcommand\hlblue[1]{#1}
\renewcommand\hlyellow[1]{#1}
\renewcommand\hlorange[1]{#1}
\renewcommand\hlgray[1]{#1}

% Underline
\renewcommand\ulred[1]{#1}
\renewcommand\ulgreen[1]{#1}
\renewcommand\ulblue[1]{#1}
\renewcommand\ulyellow[1]{#1}
\renewcommand\ulorange[1]{#1}
\renewcommand\ulgray[1]{#1}

% To hide todonotes, use: \usepackage[disable]{todonotes} in preamble.tex
% To hide changes, use: \usepackage[final]{changes} in preamble.tex


% \author{Upal Bhattacharya}
\date{}
\title{Notes for ``A Short Review for Ontology Learning: Stride to Large Language Models Trend''}
\begin{document}

\maketitle

\begingroup
    \hypersetup{linkcolor=black}
    \tableofcontents
    \pagebreak
\endgroup

\linenumbers

\section{Summary}
\label{sec:summary}

The review paper provides an outline of ontology learning (OL) approaches with an
emphasis on how LLMs are applied for OL. It outlines the advantages and
disadvantages of shallow and deep learning approaches to expound the benefits
of using large pre-trained LLMs that appear to possess superior linguistic
understanding but fail to outline the main disadvantage of LLMs: their
stochastic nature (despite available parameters to facilitate deterministic
behaviour) and the inability to `re-train' to correct for errors or, simply:
the reliance on `prompts' which significantly reduces the level of control
available.

It attempts to map various OL tasks to their equivalent NLP methods
but is inconsistent in this endeavour, often providing only a short list of
references without trying to create method-based categorizations. This is most
evident in the sections outlining LLM-based approaches for concept formation
and relationship extraction. The concept formation section can largely be
summarized as `trying different prompting approaches' with the latter
(relation extraction) being given more attention but without sufficient effort
being put into creating meaningful categorizations. As a short review, the
number of references for different method categorizations is insufficient.

Overall, the review provides an idea of what kind of work is done for
LLM-based ontology learning. The categorization using the ontology layer cake
of concept formulation and relationship extraction is helpful but there is no
attempt to categorize methodologies making it mostly a reference point for
finding other method papers.

\section{Brief Notes}
\label{sec:brief}

\begin{itemize}
  \item Presents a brief (not-rigorous) overview of broad classification of
    methods for ontolog learning, from shallow/manual methods, deep learning
    strategies and `LLM' strateges.
  \item Highlights the advantages and pitfalls of each broad stroke category
    \begin{itemize}
    \item \textbf{Shallow Methods}: \newline
      \textit{Positives}: Easy to understand and implement \newline
      \textit{Negatives}: Scalability, unable to handle complex relationships
    \item \textbf{Deep Learning Methods}: \newline
      \textit{Positives}: Better understanding of complex relationships \newline
      \textit{Negatives}: Requires large amounts of annotated data for good
      performance
    \item \textbf{LLM Methods}:
      \textit{Positives}: Superior understanding of semantics, context and
      complex relationships \newline
      \textit{Negatives}: N/A
      \begin{negative}{Missing negatives for LLMs}{neg-1}
        Does not highlight the negatives of LLMs: the inherent stochastic nature
        or lack of precise control over outputs
      \end{negative}
    \end{itemize}
  \item Categorizes OL methods as manual semi-automatic and
    fully-automatic. Manual methods, despite their simplicity, can be used as
    pre-cursor steps when using automatic methods. Semi-automatic methods,
    with some degree of human intervention might be the most effective. List
    of some semi-automatic frameworks: Text2Onto, OntoGen, OntoStudio
  \item Based on the ontology-layer cake, classifies the main tasks for OL (in
    order) as: \textbf{term extraction, synonym extraction, concept formation,
      taxonomic relation extraction, non-taxonomic relation extraction} and
    \textbf{axiom extraction}.
  \item For the deep learning and LLM-based methods, focuses on the broader
    classifications of \textbf{concept extraction} (which involves term
    extraction, synonym extraction and concept formation) and
    \textbf{relationship extraction} (which involves taxonomic and
    non-taxonomic relation extraction)
  \item Maps OL tasks to equivalent NLP tasks while highlighting relevant
    literature:
    \begin{itemize}
      \item \textbf{Deep Learning Methods}
       \begin{itemize}
       \item \textit{Concept Formation}: Named-entity recognition (NER), part-of-speech
         tagging (POS), Semantic Role Labelling, Distributed embedding similarity
       \item \textit{Relationship Extraction}: ?
        \begin{negative}{NLP-oriented classification of Relationship
            Extraction}{neg-2}
          Does not provide a clear categorization of NLP-specific tasks that map
          into relationship extraction. Only provides different architectural implementations
        \end{negative}
      \end{itemize}
    \item \textbf{LLM-based Methods}
      \begin{itemize}
        \item \textit{Concept Formation}: Natural language to OWL Funcational
          Syntax conversion, various prompting methods, [MASK]-based prompting
          for type prediction
          \begin{negative}{Haphazard Categorization of LLM-based Concept
              Formation for OL}{neg-3}
            There is no clear categorization of LLM-based tasks for Concept
            Formation. Highlights some approaches but not very thorough.
          \end{negative}
        \item \textit{Relation Extraction}:
          \begin{itemize}
          \item \textit{Hierarchical relation}: two-module methods for scoring
            relations with tree maximization objective, zero and n-shot
            prompting methods, layer-by-layer chained prompting, repeated
            prompting, annotation generation
            \begin{negative}{Lack of methodology categorizaton}{neg-4}
              Provides a review of some available methods but makes no effort
              to have categorization by methodology: e.g. prompting-focused
              methods, support-system-based methods, etc.
            \end{negative}

          \item \textit{Non-hierarchical relation}:
            \begin{positive}{Insufficient work on non-hierarchical
                relation}{pos-1}
              Acknowledges lack of sufficient work in non-hierarchical
              relation extraction
            \end{positive}
          \end{itemize}
        \end{itemize}
     \end{itemize}
   \item Future Directions: benchmarks, non-taxonomic relation extraction
     \begin{positive}{LLM-oriented evaluation benchmarks}{pos-2}
       Identifies that evaluation focusing on LLMs is a research gap. Aligns
       with our own work.
     \end{positive}
  \end{itemize}

\bibliographystyle{splncs04nat}
\bibliography{bibliography}

\pagebreak
\nolinenumbers
\appendix

\begingroup
    \hypersetup{linkcolor=black}
    \listoftodos
    \listofchanges
    \pagebreak
\endgroup

\end{document}

